%%%%%%%%%%%%%%%%%%%%%%%%%%%%%%%%%%%%%%%%%
% Jacobs Landscape Poster
% LaTeX Template
% Version 1.0 (29/03/13)
%
% Created by:
% Computational Physics and Biophysics Group, Jacobs University
% https://teamwork.jacobs-university.de:8443/confluence/display/CoPandBiG/LaTeX+Poster
% 
% Further modified by:
% Nathaniel Johnston (nathaniel@njohnston.ca)
%
% This template has been downloaded from:
% http://www.LaTeXTemplates.com
%
% License:
% CC BY-NC-SA 3.0 (http://creativecommons.org/licenses/by-nc-sa/3.0/)
%
%%%%%%%%%%%%%%%%%%%%%%%%%%%%%%%%%%%%%%%%%

%----------------------------------------------------------------------------------------
%	PACKAGES AND OTHER DOCUMENT CONFIGURATIONS
%----------------------------------------------------------------------------------------

\documentclass[final]{beamer}

\usepackage[scale=1.24]{beamerposter} % Use the beamerposter package for laying out the poster
\usepackage{graphicx}

\usetheme{confposter} % Use the confposter theme supplied with this template

\setbeamercolor{block title}{fg=ngreen,bg=white} % Colors of the block titles
\setbeamercolor{block body}{fg=black,bg=white} % Colors of the body of blocks
\setbeamercolor{block alerted title}{fg=white,bg=dblue!70} % Colors of the highlighted block titles
\setbeamercolor{block alerted body}{fg=black,bg=dblue!10} % Colors of the body of highlighted blocks
% Many more colors are available for use in beamerthemeconfposter.sty

%-----------------------------------------------------------
% Define the column widths and overall poster size
% To set effective sepwid, onecolwid and twocolwid values, first choose how many columns you want and how much separation you want between columns
% In this template, the separation width chosen is 0.024 of the paper width and a 4-column layout
% onecolwid should therefore be (1-(# of columns+1)*sepwid)/# of columns e.g. (1-(4+1)*0.024)/4 = 0.22
% Set twocolwid to be (2*onecolwid)+sepwid = 0.464
% Set threecolwid to be (3*onecolwid)+2*sepwid = 0.708

\newlength{\sepwid}
\newlength{\onecolwid}
\newlength{\twocolwid}
\newlength{\threecolwid}
\setlength{\paperwidth}{48in} % A0 width: 46.8in
\setlength{\paperheight}{36in} % A0 height: 33.1in
\setlength{\sepwid}{0.024\paperwidth} % Separation width (white space) between columns
\setlength{\onecolwid}{0.22\paperwidth} % Width of one column
\setlength{\twocolwid}{0.464\paperwidth} % Width of two columns
\setlength{\threecolwid}{0.708\paperwidth} % Width of three columns
\setlength{\topmargin}{-0.5in} % Reduce the top margin size
%-----------------------------------------------------------

\usepackage{graphicx}  % Required for including images

\usepackage{booktabs} % Top and bottom rules for tables

%----------------------------------------------------------------------------------------
%	TITLE SECTION 
%----------------------------------------------------------------------------------------

\title{Sandpiles} % Poster title

\author{Alex Pan} % Author(s)

\institute{Reed College} % Institution(s)

%----------------------------------------------------------------------------------------

\begin{document}

\addtobeamertemplate{block end}{}{\vspace*{2ex}} % White space under blocks
\addtobeamertemplate{block alerted end}{}{\vspace*{2ex}} % White space under highlighted (alert) blocks

\setlength{\belowcaptionskip}{2ex} % White space under figures
\setlength\belowdisplayshortskip{2ex} % White space under equations

\begin{frame}[t] % The whole poster is enclosed in one beamer frame

\begin{columns}[t] % The whole poster consists of three major columns, the second of which is split into two columns twice - the [t] option aligns each column's content to the top

\begin{column}{\sepwid}\end{column} % Empty spacer column

\begin{column}{\onecolwid} % The first column

%----------------------------------------------------------------------------------------
%	OBJECTIVES
%----------------------------------------------------------------------------------------

\begin{alertblock}{Abstract}

In the following, I will introduce the sandpile model and the closely-related phenomenon of self-organized criticality. I present my own simulations of some 1D and 2D sandpile configurations and discuss the results. The focus of these simulations was to investigate the presence of power laws in 2D sandpiles with varied toppling processes. Simulation results suggest that the power laws exhibited by avalanche size are invariant to dependence on neighboring states in the toppling process. 

\end{alertblock}

%----------------------------------------------------------------------------------------
%	INTRODUCTION
%----------------------------------------------------------------------------------------

\begin{block}{Introduction}

The \textit{sandpile} model was first introduced in 1978 by Bak, Tang, and Weisenfeld (BTW),  as a system capable of displaying spatial and temporal power laws and scale-invariance without finely-tuned parameters. This behavior was described as \textit{self-organized criticality} (SOC) because of the way these systems can arrive at critical states on their own. SOC was given as the underlying process behind ubiquitous and complex phenomena like fractal geometry and ``$1/f$'' noise in the 1987 paper, and has since been used to model physical systems like earthquakes and rainfall, as well as non-physical systems in sociology and stock-markets.

\end{block}

%------------------------------------------------

\begin{block}{Criticality}
\textit{Critical points} exhibit scale-invariance through features like fractals and power laws, but cannot necessarily be reached depending on a system's parameters. For example, in the Ising model, the system doesn't display critical behavior unless the parameters have been carefully tuned. However, in systems that exhibit SOC, parameters can be changed widely without affecting the systems' ability to start exhibiting critical behavior.
\end{block}

%----------------------------------------------------------------------------------------

\end{column} % End of the first column

\begin{column}{\sepwid}\end{column} % Empty spacer column

\begin{column}{\onecolwid} % The first column

%----------------------------------------------------------------------------------------
%	SANDPILES
%----------------------------------------------------------------------------------------

\begin{block}{Sandpiles}

In very general terms, sandpiles can be thought of as cellular automota. There are three things that characterize the toppling behavior of a specific configuration:
\begin{itemize}
\item A measure $z$ (not in the formal sense)
\item A critical value $K$
\item Some action $f$ by which a cell reduces its measure
\end{itemize} 
Whenever a cell's measure exceeds the critical value, it performs the action $f$. (i.e. when a cell $x$ has $z(x)>K$, $f(x)$ happens).

In the original paper, this is defined more strictly: when a cell's slope with its neighbors exceeds a critical value, it loses some of its weight to its neighbors. However, I prefer to use looser defintions because I believe stricter ones (like the aforementioned) might give some false intuition about the measure. While slope is a good example because of its physical parallels, it suggests that the measure must be defined on a relationship between a cell and its neighbors. However this does not seem to be the case according to simulated results, as I will discuss later.

For the sake of simplicity and clarity, I focused on the 1D and 2D models used by Bak, Tang, and Weisenfeld, as well as some variants in the 2D case.

\end{block}

%----------------------------------------------------------------------------------------
%	SETUP
%----------------------------------------------------------------------------------------

\begin{block}{Setup}

As physical experiments have found, observing SOC on real sandpiles is not easy or necessarily possible. Here, I use Monte Carlo simulations because they provide very high control over the parameters, which we expect to be important in observing SOC, and because they provide can provide fine-grain measurements that might not be easy to obtain in a physical setting.

In all configurations, I add grains to the system uniformly, because that seems to be the ``most'' random way of perturbing the system. The key value of interest is the distribution of avalanche sizes $P(s)$, which is calculated by tracking avalanche sizes over a sandpile's evolution. 


\end{block}

%----------------------------------------------------------------------------------------

\end{column} % End of column 2.2

\begin{column}{\sepwid}\end{column} % Empty spacer column

%----------------------------------------------------------------------------------------


\begin{column}{\onecolwid} % The first column within column 2 (column 2.1)

%----------------------------------------------------------------------------------------
%	1D BTW
%----------------------------------------------------------------------------------------
\begin{block}{1D BTW Criticality}

The 1D case serves mainly as a proof of concept and a clean example of what criticality would look like intuitively. The BTW paper bounds their 1D sandpile on the left so that particles can only leave through the right. We expect the sandpile to reach a state where adding a subsequent grain could trigger either a local or global avalanche. This is confirmed by simulation: when starting with an empty pile and uniformly adding grains, the sandpile converges to the critical state (see Figure 1).

\begin{figure}
\includegraphics[width=0.4\linewidth]{1D.jpg}
\caption{Bounded 1D sandpile after 50 000 iterations}
\end{figure}
\end{block}


%----------------------------------------------------------------------------------------
%	2D BTW
%----------------------------------------------------------------------------------------

\begin{block}{2D Configuration Results}

\begin{figure}
\includegraphics[width=0.8\linewidth]{2D-btw.jpg}
\caption{BTW Configuration}
\end{figure}

\begin{figure}
\includegraphics[width=0.8\linewidth]{2D-sloped.jpg}
\caption{``Sloped'' Configuration}
\end{figure}

\begin{figure}
\includegraphics[width=0.8\linewidth]{2D-indep.jpg}
\caption{``Independent'' Configuration}
\end{figure}

\end{block}

%----------------------------------------------------------------------------------------

\end{column} % End of column 2.2

\begin{column}{\sepwid}\end{column} % Empty spacer column

\begin{column}{\onecolwid} % The third column

%----------------------------------------------------------------------------------------
%	2D Results Continuted
%----------------------------------------------------------------------------------------

\begin{block}{2D Results continued}

Data was gathered from size 15x15 piles over 10,000 iterations, with varied measure and action in each pile. I plotted the distribution of avalanche sizes $P(s)$ and found that power laws were exhibited in all three configurations. Figures 2,3, and 4, also have the pile configurations after 10,000 iterations, which agree with the respective intuitive critical states according to their toppling rules.

\end{block}

%----------------------------------------------------------------------------------------
%	CONCLUSION
%----------------------------------------------------------------------------------------

\begin{block}{Conclusion and Future Work}

Though it is tempting to say SOC is invariant to the toppling process, that is not necessarily the case. Though invariance was observed for $P(s)$, it was found that in the ``Sloped'' configuration, plots of the avalanche length no longer exhibit power laws.
It is also noted that the toppling process does affect the range of possible avalanche sizes, even if power laws were still observed. A question that remains unanswered is whether this behavior is true for all toppling rules or if the ones I used were simply lucky picks.
Further work could be done on either a closer examination of other SOC features affected by toppling behavior, or an investigation into the effects of other topplings rules on avalanche size.

\end{block}


%----------------------------------------------------------------------------------------
%	REFERENCES
%----------------------------------------------------------------------------------------

\begin{block}{Selected References}

\nocite{*} % Insert publications even if they are not cited in the poster
\small{\bibliographystyle{unsrt}
\bibliography{sample}\vspace{0.75in}}

\end{block}


%----------------------------------------------------------------------------------------
%	CONTACT INFORMATION
%----------------------------------------------------------------------------------------

\setbeamercolor{block alerted title}{fg=black,bg=norange} % Change the alert block title colors
\setbeamercolor{block alerted body}{fg=black,bg=white} % Change the alert block body colors

\begin{alertblock}{Simulations}

\begin{itemize}
\item Github: \href{https://github.com/apanana/Sandpiles}{https://github.com/apanana/Sandpiles}
\end{itemize}

There are several Mathematica notebooks for data collection and user-interaction modules for visualization
\end{alertblock}

%----------------------------------------------------------------------------------------

\end{column} % End of the third column

\end{columns} % End of all the columns in the poster

\end{frame} % End of the enclosing frame

\end{document}
